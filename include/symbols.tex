% Credits: Stephen Boyles
% https://sboyles.github.io
% Modified by: Debojjal Bagchi
% https://debojjalb.github.io

%%%%% Assorted macros and commands

%% Author and other details related
\newcommand{\Keywords}[1]{\par\noindent
{\small{\em \textbf{Keywords}\/}: #1}}

\newcommand\blfootnote[1]{%
  \begingroup
  \renewcommand\thefootnote{}\footnote{#1}%
  \addtocounter{footnote}{-1}%
  \endgroup
}

% Math related
\newcommand{\mb}[1]{\mathbf{#1}} % Abbreviation for bold math
\newcommand{\mr}[1]{\mathrm{#1}} % Abbreviation for regular text in equations
\newcommand{\mc}[1]{\mathcal{#1}} % Abbreviation for script math
\newcommand{\mbb}[1]{\mathbb{#1}} % Abbreviation for blackboard bold
\newcommand{\bbr}{\mbb{R}} % Abbreviation for set of real numbers
\newcommand{\bbc}{\mbb{C}} % Abbreviation for set of complex numbers
\newcommand{\bbz}{\mbb{Z}} % Abbreviation for set of integers
\newcommand{\mathcalb}[1]{{\boldsymbol{\mathcal{#1}}}} % Abbreviation of bold math symbol

\newcommand{\myp}[1]{\left( #1 \right)} % Correct-sized parentheses () 
\newcommand{\mys}[1]{\left[ #1 \right]} % Correct-sized square brackets []
\newcommand{\myc}[1]{\left\{ #1 \right\}} % Correct-sized curly brackets {}

\newcommand{\pdr}[2]{\frac{\partial #1}{\partial #2}} % Partial derivatives
\newcommand{\pdrb}[2]{\frac{\partial^2 #1}{\partial #2^2}} % Second partial derivative
\newcommand{\pdrc}[3]{\frac{\partial^2 #1}{\partial #2 \partial #3}} % Mixed second partial derivative
\newcommand{\ttt}{\texttt} % Typewriter font in regular text
\newcommand{\ds}{\displaystyle} % Forces full equation size in regular text
\newcommand{\vect}[1]{\begin{bmatrix} #1 \end{bmatrix}} % Create vectors and matrices
\newcommand{\minivect}[1]{\mys{\begin{matrix} #1 \end{matrix}}} % Create vectors and matrices
\newcommand{\labeleqn}[2]{\begin{equation} #2 \label{eqn:#1} \end{equation}} % Create an equation with a label
\newcommand{\nolabeleqn}[1]{\begin{equation} #1 \end{equation}} % Create an equation with no label
\newcommand{\eqn}[1]{\eqref{eqn:#1}} % Refer to an equation
\newcommand{\genfig}[4]{\begin{figure} \centering \includegraphics[#4]{#2} \caption{#3 \label{fig:#1}} \end{figure}}
\newcommand{\stevefig}[3]{\begin{figure} \centering \includegraphics[width=#3]{#1.pdf} \caption{#2 \label{fig:#1}} \end{figure}}
\newcommand{\citeapos}[1]{\citeauthor{#1}'s (\citeyear{#1})}
\newcommand{\citeaposint}[2]{\citeauthor{#1}'s {#2} (\citeyear{#1})}
\renewcommand{\solution}[1]{\textbf{Solution.} #1 $\blacksquare$}
\newcommand{\optimize}[1]{\[ \begin{array}{>{\ds}r>{\ds}l>{\ds}l} #1 \end{array} \]} % Creates a formatted optimization problem

% For algorithm creates input/output
\algrenewcommand\algorithmicrequire{\textbf{Input:}}
\algrenewcommand\algorithmicensure{\textbf{Output:}}
\DeclareMathOperator*{\argmax}{arg\,max}

% coloured texts
\newcommand{\orange}[1]{{\color{myorange}#1}}

% custom spacing
\newcommand{\vs}{\vspace{5mm}}